%------------------------------------------------------------------------
%% Document packages
%------------------------------------------------------------------------
\usepackage[english]{babel}
\usepackage[utf8]{inputenc}
\usepackage{amsfonts}
\usepackage[T1]{fontenc}
\usepackage{fancyhdr}
\usepackage[sort, authoryear]{natbib}

% needed for custom colors
\usepackage[dvipsnames]{xcolor} % needs to be declared before tikz or pstricks package
\usepackage{tikz}

% package needed for "clickable" linked table of content
\usepackage[pdfencoding=auto]{hyperref}

% removing paragraph indents
\usepackage{parskip}

\usepackage{amsmath} % needed for text in equations

% suppress pdf group warnings
\pdfsuppresswarningpagegroup=1

% don't strech page content from top to bottom
\raggedbottom

%
\usepackage{minitoc}
%------------------------------------------------------------------------
% Usefull commands
%------------------------------------------------------------------------

\definecolor{lightBlue}{HTML}{268bd2}

% message command -> that we hide in the final version
\newcommand{\chapterMessage}[1]{
    \fcolorbox{lightBlue}{white}{\begin{minipage}{40em}
    {\color{lightBlue} Message:}
              #1
    \end{minipage}}
}

\usepackage{ifthen}

\newcommand{\eq}[2][none]{
\begin{equation}
#2
\ifthenelse{
    \equal{#1}{none}
    }{
    % no label specified
    }{
    \label{#1}
    }
\end{equation}
}

\newcommand{\addChapter}[1]{
    \input{./chapters/#1}
    \pagebreak
}

% todo if this is the first occurence of this abberivation print full text (aberivation) otherwise print abberivation
% Finate Differences Method
% FEM, BEM
\newcommand{\abberivation}[1]{
    #1
}

\newcommand{\q}[2]{
    \textbf{Question:} #1\\
    \smallskip
    #2
}

% -
% plots
% -
% for subfigure
\usepackage{subcaption}
\usepackage{wrapfig}
\usepackage{graphicx} % needed for includegraphics
\usepackage{xparse} % imports \NewDocumentCommand so we can define functions with custom parameters
\usepackage{float} %H for figure positioning

\NewDocumentCommand{\plot}{O{none} O{none} m}{
\begin{figure}[H]
\begin{center}
\begin{subfigure}[b]{0.6\textwidth}
\includegraphics[width= \textwidth]{#3}
\end{subfigure}
% adding the caption
\ifthenelse{
\equal{#1}{none}
    }{
    % no label specified
    }{
    \caption{#1}
    }

% adding the label
\ifthenelse{
\equal{#2}{none}
    }{
    % no label specified
    }{
    \label{#2}
    }
\end{center}
\end{figure}
}

\NewDocumentCommand{\dualPlot}{O{none} O{none} m m}{
\begin{figure}[H]
\begin{center}
\begin{subfigure}[b]{0.48\textwidth}
\includegraphics[width= \textwidth]{#3}
\end{subfigure}
\begin{subfigure}[b]{0.48\textwidth}
\includegraphics[width= \textwidth]{#4}
\end{subfigure}

% adding the caption
\ifthenelse{
\equal{#1}{none}
    }{
    % no label specified
    }{
    \caption{#1}
    }

% adding the label
\ifthenelse{
\equal{#2}{none}
    }{
    % no label specified
    }{
    \label{#2}
    }
\end{center}
\end{figure}
}

\NewDocumentCommand{\triplePlot}{O{none} O{none} m m m}{
\begin{figure}[H]
\begin{center}
\begin{subfigure}[b]{0.32\textwidth}
\includegraphics[width= \textwidth]{#3}
\end{subfigure}
\begin{subfigure}[b]{0.32\textwidth}
\includegraphics[width= \textwidth]{#4}
\end{subfigure}
\begin{subfigure}[b]{0.32\textwidth}
\includegraphics[width= \textwidth]{#5}
\end{subfigure}

% adding the caption
\ifthenelse{
\equal{#1}{none}
    }{
    % no label specified
    }{
    \caption{#1}
    }

% adding the label
\ifthenelse{
\equal{#2}{none}
    }{
    % no label specified
    }{
    \label{#2}
    }
\end{center}
\end{figure}
}

\NewDocumentCommand{\quarterPlot}{O{none} O{none}  m m m m}{
\begin{figure}[H]
\begin{center}
\begin{subfigure}[b]{0.24\textwidth}
\includegraphics[width= \textwidth]{#3}
\end{subfigure}
\begin{subfigure}[b]{0.24\textwidth}
\includegraphics[width= \textwidth]{#4}
\end{subfigure}
\begin{subfigure}[b]{0.24\textwidth}
\includegraphics[width= \textwidth]{#5}
\end{subfigure}
\begin{subfigure}[b]{0.24\textwidth}
\includegraphics[width= \textwidth]{#6}
\end{subfigure}

% adding the caption
\ifthenelse{
\equal{#1}{none}
    }{
    % no label specified
    }{
    \caption{#1}
    }

% adding the label
\ifthenelse{
\equal{#2}{none}
    }{
    % no label specified
    }{
    \label{#2}
    }
\end{center}
\end{figure}
}

% ------------------------------------------------------------
% code highlithing: xcoler and listings needed
% ------------------------------------------------------------

\definecolor{eminence}{RGB}{108,48,130}
\definecolor{weborange}{RGB}{255,140,0}
\definecolor{lightBlue}{HTML}{268bd2}
\definecolor{lightBlack}{HTML}{585858}
\definecolor{lightRed}{HTML}{D00000}
\definecolor{lightGreen}{HTML}{20B2AA}

\usepackage{listings}

\lstset {
    language=C++,
    tabsize=4,
    showstringspaces=false,
    commentstyle=\color{lightGreen},
    keywordstyle=\color{eminence},
    stringstyle=\color{lightRed},
    basicstyle=\small\ttfamily\color{lightBlack}, % basic font setting
    emph={int,char,double,float,unsigned,void,bool},
    emphstyle={\color{lightBlue}},
    % keyword highlighting
    classoffset=1, % starting new class
    otherkeywords={>,<,.,;,-,!,=,~},
    morekeywords={>,<,.,;,-,!,=,~},
    keywordstyle=\color{weborange},
    classoffset=0,
    xleftmargin=.1\textwidth,
}

\lstnewenvironment{code}[1]{
\renewcommand\lstlistingname{Code}%
\lstset{caption={#1}}
}%
{}

\usepackage{tcolorbox}

\usepackage[T1]{fontenc}
\usepackage{lmodern}

\newtcbox{\cw}{nobeforeafter,
colframe=lightBlue!10!white,
colback=lightBlue!10!white,
boxrule=0.5pt,
arc=3pt,
fontupper=\color{lightBlue}\ttfamily\bfseries,
boxsep=0pt,
left=2pt,
right=2pt,
top=3pt,
bottom=3pt,
tcbox raise base}

\newcommand{\todo}[1]{
    \cw{todo}: #1
}
